\documentclass{article}

\topmargin -.5in
\textheight 8.25in
\oddsidemargin 0in
\evensidemargin 0in
\textwidth 6.5in
\parindent 3em

\usepackage{amsmath}

%% Please use the following statements for
%% managing the text and math fonts for your papers:
\usepackage{times}
%\usepackage[cmbold]{mathtime}
\usepackage{bm}
\usepackage{natbib}

%\usepackage{algorithm}
%\usepackage{fancyhdr}
\usepackage{graphics,color,pict2e}
\usepackage{epsfig}
%\usepackage{times}
%\usepackage[cmbold]{mathtime}
%\usepackage{bm}
%\usepackage{rotating}
%\usepackage{pdflscape}
%\usepackage{longtable}



%\usepackage{C:/R/R-2.11.1/share/texmf/Sweave}

%\theoremstyle{plain}
%\newtheorem{Theorem}{Theorem}[section]
\newtheorem{theorem}{Theorem}
\newtheorem{corrolary}{Corrolary}
\newtheorem{result}{result}
\newtheorem{Lemma}{Lemma}
\newcommand{\toi}{t}
\newcommand{\eqd}{\buildrel D \over =}
\newcommand{\vect}{\bf }
\newcommand{\sle}{\stackrel{st}{\le}}




\newcommand{\ch}[2]{\ensuremath{ \left( \begin{array}{c} #1 \\ #2 \end{array} \right) }}
\newcommand{\dfn}[2]{\ensuremath{ \left( #1 \right)_{(#2)}}}



\begin{document}

{\bf \large Notes on: Prevalence Estimates from Surveys with Imperfect Assays}

\today


%\begin{abstract}
%asdf
%\end{abstract}

Damon,

Here are some notes on the project. You can change the notation if you want. I am just putting down something so we can be clear about what we are doing.

\section{Overview}

Quick overview. I described the problem in Section~\ref{sec-math}. The usual adjustment for sensitivity and specificity is described in Section~\ref{sec-usualAdj}. Then the three parts I spoke about. First, the confidence interval for a simple random sample is described in Section~\ref{sec-thetastar} (that is what you already did). Second, the confidence interval assuming a perfect assay in Section~\ref{sec-betaPerfect}. And finally, the 
confidence interval for an imperfect assay in Section~\ref{sec-betaImperfect}. 


\section{Mathematical Statement of Problem}
\label{sec-math}

Suppose we take a survey, that is divided into $k$ sampling units (e.g., states in the US).
We take a simple random sample of individuals from each unit, perform and assay on each individual to determine who has a disease, and measure the number of individuals who have a positive value for the assay.

Let $X_i$ be the number of positives by the assay, out of $n_i$ measured in sampling unit $i$.
Suppose we had a perfect assay with 100\% sensitivity and 100\% specificity. Then let $X_i^*$ be the true number with the disease out of the $n_i$ sampled in unit $i$.

Let $X_i^* \sim Binomial(n_i, \theta_i^*)$ for $i=1,\ldots,k$. We are interested in the prevalence of disease in the population. Suppose there are $N_i$ individuals in the population of unit $i$. Then the prevalence of the population is
\begin{eqnarray*}
\beta & = & \frac{ \sum_{i=1}^{k} N_i \theta_i^* }{ \sum_{j=1}^{k} N_j }  =  \sum_{i=1}^{k} w_i \theta_i,
\end{eqnarray*}
where $w_i = N_i/N$ and $N=\sum_{j=1}^{k} N_j$.


Suppose we evaluate the assay. We measure the assay on $m_n$ individuals known not to have the disease (negative controls), and on $m_p$ individuals known to have the disease (positive controls). Let $C_n$ and $C_p$ be the number who tested positive out of the $m_n$ in the negative control and the $m_p$  in the positive control. Now make a convenience assumption:
\begin{itemize}
\item Assume that the negative and positive controls act like simple random samples from their populations (e.g., the $m_n$ negative controls are like a simple random sample from the total population of individuals without the disease).
\end{itemize}
Under that assumption, $C_n \sim Binomial(m_n, \phi_n)$ and $C_p \sim Binomial(m_p, \phi_p)$,
where $1-\phi_n$ is the specificity (true negative rate of the assay),
and $\phi_p$ (true positive rate of the assay) is the sensitivity.

Here is the statistical problem. Assuming all of the above holds, and we observe realizations of $X_1,\ldots, X_k$ and $C_n$ and $C_p$, and the other values are known constants ($n_1,\ldots, n_k$, $w_1,\ldots, w_k$,  $m_n$, and $m_p$), what is a good estimate and confidence interval procedure for $\beta$.


\section{Usual Adjustment for Sensitivity and Specificity}
\label{sec-usualAdj}

To simplify the problem, suppose there is only one sampling unit, and let $X_1=X$ and $n_1=n$, etc. Then $\beta = \theta_1^*=\theta^*$. 
Further assume that $\phi_p$ and $\phi_n$ are known. Let $Y_1,\ldots,Y_n$ be the result of the assay from the $n$ individuals in the population
($Y_i=1$ is positive and $Y_i=0$ is negative). Let $Y_1^*,\ldots,Y_n^*$ be the true disease status ($Y_i^*=1$ is diseased, and $Y_i^*=0$ is not diseased).  
Then using the definition of sensitivity and specificity, we get:
\begin{eqnarray*}
Pr[ Y_i =1 | Y_i^*=1] & = & \phi_p \\
Pr[ Y_i=0 | Y_i^*=0] & = & 1- \phi_n 
\end{eqnarray*}
and 
\begin{eqnarray*}
Pr[ Y_i=1] & = & Pr[ Y_i =1 | Y_i^*=1]  Pr[Y_i^*=1] + Pr[ Y_i =1 | Y_i^*=0] Pr[ Y_i^*=0] \\
& = & \phi_p \theta^* + \phi_n (1-\theta^*)  \\
& = & \theta^* (\phi_p - \phi_n) + \phi_n.
\end{eqnarray*}
Let $\theta = Pr[ Y_i=1]$.

So we have 
\begin{eqnarray*}
& & X \sim Binomial \left(n, \theta \right)  \\
\mbox{ where} & \theta & = \theta^* \left\{ \phi_p - \phi_n \right\} + \phi_n.
\end{eqnarray*}
Thus, 
\begin{eqnarray}
\theta^* & =  & \frac{ \theta - \phi_n }{ \phi_p - \phi_n}  = \frac{ \theta + (1 - \phi_n) -1  }{ \phi_p + (1 - \phi_n) -1 } \label{eq:thetastar}
\end{eqnarray}

Suppose we estimate the sensitivity as $\widehat{Se} = \hat{\phi}_p = C_p/m_p$ and the specificity as $\widehat{Sp} = 1-\hat{\phi}_n = 1- C_n/m_n$,
and we define the apparent prevalence as the sample proportion of the positive assay results, $\hat{\theta} = x/n$.
This gives the usual estimator for  prevalence adjusted for sensitivity and specificity
 \citep[see e.g.,][]{Roga:1978},
\begin{eqnarray*}
\hat{\theta}^* & = & \frac{ \hat{\theta} - \hat{\phi}_n }{\hat{\phi}_p - \hat{\phi}_n} = \frac{ \hat{\theta} + \widehat{Sp} -1 }{\widehat{Se} + \widehat{Sp} -1}.
\end{eqnarray*}
 
 
\section{Confidence Interval on $\theta^*$}
\label{sec-thetastar}

The simulation you (Damon) already did handles this problem. It answers: what is a good confidence interval for $\theta^*$?
You compared the method of \citet{Lang:2014} to the new method described in the Supplement to \citet{Kali:2021}, using the method of \citet{FayP:2015}.
It shows that the lower error has a better bound with the new method. 

\section{Confidence Interval on $\beta$ with a Perfect Assay}
\label{sec-betaPerfect}


Suppose we have a perfect assay (i.e., $\phi_p=1$ and $\phi_n=0$). Then $\theta^* = \theta$, and $\beta = \sum_{i=1}^{k} w_i \theta$. 
The problem reduces to getting a confidence interval for a weighted average of the means of sample proportions. 
This problem is a standard survey sampling one. Some methods are \citet{Korn:1998}, see the review in \citet{Dean:2015}. 

Another way to address this problem is to assume that $X_i \sim Poisson( n_i \theta_i)$ and use the methods 
of \citet{FayF:1997} or \citet{FayK:2017}. This should give a confidence interval for $\beta=\sum_{i=1}^{k} w_i \theta_i$ 
that gives a lower and upper confidence distribution that is distributed gamma. (I will let you work out those details. Let me know if it is not easy.)  


\section{Confidence Interval on $\beta$ with an Imperfect Assay}
\label{sec-betaImperfect}

This is the final problem (described in Section~\ref{sec-math} that combines the two other problems.

We have 
\begin{eqnarray*}
\beta & = & \sum_{i=1}^{k} w_i \theta_i^*  \\
& = &  \sum_{i=1}^{k} w_i  \frac{ \theta_i - \phi_n }{ \phi_p - \phi_n} \\
& = &   \frac{ \sum_{i=1}^{k} w_i  \theta_i  }{ \phi_p - \phi_n} - \frac{ \phi_n \sum_{i=1}^{k} w_i    }{ \phi_p - \phi_n} \\
& = &   \frac{ \sum_{i=1}^{k} w_i  \theta_i  }{ \phi_p - \phi_n} - \frac{ \phi_n     }{ \phi_p - \phi_n},
\end{eqnarray*}
where the last step is because the weights are assumed to sum to 1. 

We can use 
\begin{eqnarray*}
\hat{\beta} & = &   \frac{ \sum_{i=1}^{k} w_i  \hat{\theta}_i  }{ \hat{\phi}_p - \hat{\phi}_n} - \frac{ \hat{\phi}_n   }{ \hat{\phi}_p - \hat{\phi}_n} \\
& = & 
\end{eqnarray*}
as our estimator of $\beta$. Then we can use the melding method of \citet{FayP:2015} to combine the confidence distributions for $\sum_{i=1}^{k} w_i  {\theta}_i$, for $\phi_p$,
and for $\phi_n$. That might not be straightforward, since I am not sure if $\phi_n$ and $\phi_p$ are monotonic in $\beta$. You need to check that out.


We can compare that new method to the method described in the supplement of \citet{Kali:2021}.




%\bibliographystyle{agsm}
\bibliographystyle{biom}


\bibliography{refs}

\end{document}